\documentclass[letterpaper, 10pt]{article}

\usepackage{fancyhdr}
\usepackage[left=2.35cm,top=2.45cm,bottom=2.45cm,right=2.35cm,letterpaper]{geometry}
\usepackage{amsmath}

\pagestyle{fancy}
\fancyhf{}
\fancyfoot[C,CO]{\thepage}
\fancyhead[L,LO]{\bfseries EECS 301}
\fancyhead[C,CO]{\bfseries Lab 1}
\fancyhead[R,RO]{Summer 2016}

	\newenvironment{dq}{\textbf{Design Questions}}
	
	\newenvironment{q1}{\noindent Question 1:}
	
	\newenvironment{a1}{\noindent}
	
	\newenvironment{q2}{\noindent Question 2:}
	
	\newenvironment{a2}{\noindent}
	
	\newenvironment{q3}{\noindent Question 3:}
	
	\newenvironment{a3}{\noindent}
	
	\newenvironment{q4}{\noindent Question 4:}
	
	\newenvironment{a4}{\noindent}

\begin{document}
	\noindent
	Andrew Covarrubias (axc554)
	\\
	Jonathan Monreal (jem177)
	\bigskip
	\bigskip
	\\
	\begin{dq}
		
		\bigskip
		
		\begin{q1}
			\\
			\textbf{
			How many states did you use for each machine? How did you choose what these states
			were? Consider using sub-states to achieve the multi-cycle delay before transitioning out
			of a state (e.g. left\_signal\_three\_lights\_0, left\_signal\_three\_lights\_1,
			left\_signal\_three\_lights\_2).}
			
		\end{q1}
		
		\bigskip
		
		\begin{a1}
			The brake machine has 3 states in total. idle, break1, and break2 \\
			The turn signal machine has 22 states in total. IDLE, ERROR\_STATE, left with 10 sub-states, and right with 10 sub-states.\\
			The states names were given in the assignment and the sub-state names were reasonable names. 
			
		\end{a1}
		
		\bigskip
		
		\begin{q2}
			\\
			\textbf{
			How many bits did you use in the counter to generate the ~3Hz clock?}
		\end{q2}
		
		\bigskip
		
		\begin{a2}
			In order to get the clock from 50 MHz we used division by the powers of two to get 23 bits. From 2\textsuperscript{24} = 16777216, multiplying by 3 we get 50331648 which is close to 50 MHz
		\end{a2}
		
		\bigskip
		
		\begin{q3}
			\\
			\textbf{Which I/O signals that you need on the DE1-SoC development board are active-low?}
		\end{q3}
		
		\bigskip
		
		\begin{a3}
			The buttons and hex displays are the only active low components we used in this lab. 
		\end{a3}
		
		\bigskip
		
		\begin{q4}
			\\
			\textbf{How did you verify that you correctly implemented the Verilog state machine syntax in the Quartus software?}
		\end{q4}
		
		\bigskip
		
		\begin{a4}
			We were able to verify the state machine diagram by bringing up the state machine diagram feature in Quartus.
		\end{a4}
		
		\bigskip
		
		
	\end{dq}
	
	
\end{document}